\documentclass{article}
 
\begin{document}

\section{Framework of the macroalgae module}

\subsection{General description}

The macroalgae module is designed to model the the dynamics of kelp\textit{Saccharina} and is based almost entirely on the model described in Broch \textit{et al.} (2011). The general model organization will be described here, specifically the implementation in the DELWAQ library. For further details of the derivation of the model equations please refer to Broch \textit{et al.} (2011).

Macroalgae can be idealized as 4 distinct state variable components:
\begin{itemize}
\item MacroALgae Structural mass (MALS)
\item MacroALgae Nitrogen storage mass (MALN)
\item MacroALgae Phosphorous storage mass (MALP)
\item MacroALgae Carbon storage mass (MALC)
\end{itemize}
Collectively these components represent the entire macroalgae frond, where frond is the term used to describe the macro structure of the algae visible to the naked eye, including the foot. Each component exhibits its own growth dynamics and interacts with the surrounding water independently. 

For instance, the structural mass represents the part of the macroalgae that increases the volume of the total front when it grows. It does not take nutrients from the water, does not photosynthesize, and loses mass when it decays, producing detritus. It grows by taking nutrients (N/P/C) from its storage, and is thus responsible for growth respiration. The structural mass is analogous to the common notion of 'dry matter' and represents the dry weight of the plant minus the weight of water, nutrient stores, and carbon stores.
 
The nitrogen and phosphorous storage uptake nutrients (NH4, NO3, PO4) from the ambient water and store them for the structural mass during growth periods. In this model it is assumed the phosphorous storage dynamics are analagous to the nitrogen, but in reality the behaviour of phosphorous storage is not well known.

The carbon storage is the part of the plant responsible for photosynthesizing, producing exudate, and maintenence respiration. It produces carbon by taking up inorganic carbon and producing oxygen via photosynthesis. It provides carbon to the structural mass during growth periods. 

It is assumed that there is constant relationship between volume and area of a macroalgae frond. This means that the density of the macroalgae does not change, even if the ratio of structural mass to stored material changes. Additionally, the C:N and C:P ratios of each component are fixed, but because the ratio of stored mass to structural mass changes, the C:N and C:P ratios of the entire frond changes in due to changes in ambient conditions.

\subsection{Flux of Macroalgae structural biomass}
\begin{flushright}
\textsc{process: FLMALS}
\end{flushright}

The structural component of the macroalgae is the dry material that gives shape and structure to the macroalgae frond. As it grows it increases the length and area of the entire frond. Growth of other components of the frond do not have any effect on the frond length, volume, density, or surface area. 

The net growth of the structural mass is the resulting rate of structural biomass production and frond erosion (considered to be analagous to mortallity used in other DELWAQ processes). This balance can be defined by the following function:

\[dGrowMALS =MALS \times (\mu - \phi)\]
 
The growth $\mu$ rate of MALS is dependent firstly on the nutrient (N, P, and C) stores available. Its growth rate is defined as follows:

\[\mu = f_{density} f_{photoperiod} f_{temperature}\times min\big\{1-\frac{N_{min}}{MALN},1-\frac{P_{min}}{MALP},1-\frac{C_{min}}{MALC}\big\}\]

Where:
\begin{itemize}
\item $\mu$ specific growth rate of macroalgae sturctural mass (d$^{-1}$)
\item $\phi$ mortality/erosion rate (d$^{-1}$)
\item $f_{density}$ biomass density limitation function (-)
\item $f_{photoperiod}$ photoperiod limitation function (-)
\item $f_{temperature}$ temperature limitation function (-)
\item $MALS$ structural mass (gDM m$^{-2}$)
\item $N_{min}$ minimum N storage (gN gDM$^{-1}$)
\item $MALN$ nitrogen storage (gN/m$^2$)
\item $P_{min}$ minimum phosphorous storage (gP gDM$^{-1}$)
\item $MALP$ phosphorous storage (gP/m$^2$)
\item $C_{min}$ minimum carbon storage (gC gDM$^{-1}$)
\item $MALC$ carbon storage (gC/m$^2$)
\end{itemize}

Note that in the code the storage terms are converted to g/gDM using $MALS$ to comply with the formulations outlined in Broch \textit{et al.} (2011). The storage state variable terms are g/m$^2$ as per all non-transportable DELWAQ substances.

The growth rate is dependent on a density limitation, a photoperiod limitation, and a temperature limitation. The density limitation represents the ability of the frond to only grow so big, and the bigger it gets the lower its growth rate will become. The photoperiod limitation is similar to the daylength limitation used in BLOOM and DYNAMO, but instead considers the \textit{change} in daylength compared to the previous day instead of the actual current daylength. This is due to the fact that \textit{Saccharina} is a seasonal anticipator and will grow and store nutrients in accordance with the change in the season as determined by how much longer or shorter the days become. The temperature limitation is a simple piece wise function that identifies an optimal growth between temperatures 10-15$^0$C, no growth above 19$^0$C, and linear growth increasing betwen -1.8 and 10$^0$C.  The limitation functions are formulated as follows:

\[f_{density} = m_1 exp\big\{-(\frac{MALS}{MALS_0})^2 \big\}+m_2\]
\[f_{photoperiod} = a_1 (1+sin(\tau (n) | \tau (n)| ^{\frac{1}{2}})) + a_2\]

where $\tau$ is a function describing the normalized difference in day length between current day and previous day.

\[f_{temperature} =  
\left \{
  \begin{tabular}{cc}
   0.08T + 0.2                 & -1.8 \textless 10 \\
   1                           & 10 \leq T \leq 15 \\
   19/4 - T/4                  & 15 \leq T \leq 19 \\ 
   0                           & T \textgreater 19 \\
  \end{tabular}
\right \}
\]

Where:
\begin{itemize}
\item $m_1$ mortality parameter 1 (-)
\item $m_2$ mortality parameter 1 (-)
\item $MALS_0$ critical biomass area (m$^2$)
\item $a_1$ photoperiod parameter 1 (-)
\item $a_2$ photoperiod parameter 2 (-)
\end{itemize}

The specific mortality rate $mrt$ of the structural biomass is dependent on the total area of the frond in relation to a critical large area and the erosion parameter. The formulation is as follows:

\[\phi = \frac{10^{-6}exp(\epsilon\times MALS)}{1+10^{-6}exp((\epsilon\times MALS)-1)}\]

Where:
\begin{itemize}
\item $\epsilon$ erosion parameter (m$^{-2}$)
\end{itemize}

\subsection{Flux of Macroalgae nutrient storage}
\begin{flushright}
\textsc{process: FLMALN}
\end{flushright}

The nutrient storage component(s) of the macroalgae are those that the structural mass grows upon. The storage is also responsible for taking up nutrients from the ambient water. They consist of a nitrogen (MALN) and a phosphorous (MALP) storage component, which are both dealt with in the same nutrient storage subroutine. Note that Broch \textit{et al.} (2011) does not include a phosphorous component to the nutrient storage, but it is included in this model for flexibility should new information about phosphorous storage become available.

The change in a nutrient storage mass is given by the following equations:

\[dUptMALN = J_N - mu(K_c \times k_N \times MALS + MALN)\]
\[dUptMALP = J_P - mu(k_C \times k_P \times MALS + MALP)\]

This represents the balance between uptake $J$ and utilization for growth of the frond's structural mass. The uptake is dependent on flow velocity, the amount of stores compared to the minimum and maximum possible, and the ambient nutrient concentration. The formulation for both nitrogen and phosphorus is described as follows:

\[J_N = f_{velocity}J_{Nmax}(\frac{NO_3^-}{K_{sn}+NO_3^-})(\frac{MALN_{max}-MALN}{MALN_{max}-MALN_{min}})\]

\[J_P = f_{velocity}J_{Pmax}(\frac{PO_4^{3-}}{K_{sp}+PO_4^{3-}})(\frac{MALP_{max}-MALP}{MALP_{max}-MALP_{min}})\]

Where the first term pertains to the uptake of nutreitns and the second term to the utilization of stores by the structural mass. Note how the nutrient requirement is dependent on those required to add more structural mass and more storage mass. This is because the total storage in the frond will increase as the frond grows, but the ratio of storage to structural mass is maintained for the growth timestep. The velocity function $f_{velocity}$ is formulated as:

\[f_{velocity} = 1-exp(-\frac{U}{U_{0.65}})\]

and where:

\begin{itemize}
\item $k_C$        carbon:dry matter ratio in structural mass (gC gDM$^{-1}$)
\item $k_N$        nitrogen:carbon ratio in structural mass (gN gC$^{-1}$)
\item $k_P$        phosphorous:carbon ratio in structural mass (gN gC$^{-1}$)
\item $J_{Nmax}$   maximum uptake rate nitrogen (gN m$^{-2}$d$^{-1}$)
\item $K_{sn}$     half saturation for inorganic nitrogen uptake
\item $NO_3^-$     ambient nitrate concentration (gN m$^{-3}$)
\item $MALN_{max}$ maximum nitrogen storage (gN gDM$^{-1}$)
\item $MALN_{min}$ minimum nitrogen storage (gN gDM$^{-1}$)
\item $J_{Pmax}$   maximum uptake rate nitrogen gP m$^{-2}$d$^{-1}$)
\item $K_{sp}$     half saturation for inorganic nitrogen uptake (gP m$^{-3}$)
\item $PO_4^{3-}$  ambient phosphate concentration (gP m$^{-3}$)
\item $MALP_{max}$ maximum phosphorous storage (gP gDM$^{-1}$)
\item $MALN_{min}$ minimum phosphorous storage (gP gDM$^{-1}$)
\item $U$          water velocity (m s$^{-1}$)
\item $U_{0.65}$   water velocity at which uptake rate is 65% of maximum (m s$^{-1}$)
\end{itemize}

The uptake of nutrients is dependent on the ambient concentration according to Michaelis-Menten kinetics. It is dependent on velocity such that high velocities make it easier for the frond to uptake the nutrients. This is related to an improved mass transfer coefficient. At sufficiently high ambient concentrations and water velocities the uptake rate is $J_{max}$.

\subsection{Flux of Macroalgae carbon storage}
\begin{flushright}
\textsc{process: FLMALC}
\end{flushright}

The carbon storage component(s) of the macroalgae are those that the structural mass grows upon. The carbon storage (MALC)is also responsible for photosynthesis, maintenance respiration and exudation. 

The change in a carbon storage mass is given by the following equations:

\[dUptMALC = P(1-E)-R - \mu(k_C \times MALS + MALC)\]

Where:
\begin{itemize}
\item $P$ Gross photosynthetic rate (gCm$^{-3}$ d$^{-1}$)
\item $E$ Fraction exudation (-)
\item $P$ Maintenance respiration rate (gCm$^{-3}$ d$^{-1}$)

\end{itemize}

Where the first term relates to the net of production, respiration and exudation, and the second term pertains to the utilization of carbon stores by the structural mass.

The growth production rate is given by the following equation:

\[P = P_s(1-exp(-\frac{\alpha \times I}{P_s}))exp(-\frac{\beta \times I}{P_s})\]

\[P_s = \frac{\alpha \times I_{sat}}{ln(1+\frac{\alpha}{\beta})}\]

\[P_{max} = (\frac{\alpha \times I_{sat}}{ln(1+\frac{\alpha}{\beta}})(\frac{\alpha}{\alpha+\beta})(\frac{\beta}{\alpha+\beta})^{\frac{\beta}{\alpha}}\]

\[P_{max} = \frac{P_1exp(\frac{T_{AP}}{T_{P1}})}{1+exp(\frac{T_{APL}}{T} - \frac{T_{APL}}{T_{PL}})+exp(\frac{T_{APH}}{T_{PH}}-\frac{T_{APH}}{T})}\]

\begin{itemize}
\item The equations for $P_{max}$ are non-linear in the parameter $\beta$ and so the equations are solved using Newton's method in the code.
\end{itemize}

Where:

\begin{itemize}
\item $P_s$ Saturation photosynthetic rate (gC m$^{-3}$ d$^{-1}$)
\item $I$ Incident radiation (W m$^{-2}$)
\item $I_{sat}$ Saturation radiation (W m$^{-2}$)
\item $\alpha$ Photosynthetic efficiency (gC d$^{-1}$W$^{-1}$) 
\item $\beta$ Photosynthetic light inhibition (gC d$^{-1}$W$^{-1}$) 
\item $P_1$ Reference photosynthetic rate at T$_1$
\item $P_2$ Reference photosynthetic rate at T$_2$
\item $T$ Water temperature (K)
\item $T_{P1}$ temp for reference photosynthetic rate 1
\item $T_{P2}$ temp for reference photosynthetic rate 2

\item $T_{AP}$ Arrhenius temperature for photosynthesis               (degK)
\item $T_{APH}$ Arrhenius temp for photosynthesis high end             (degK)
\item $T_{APL}$ Arrhenius temp for photosynthesis low end             (degK)

\end{itemize}

\subsection{Distribution of macroalgae biomass in the water column}
\begin{flushright}
\textsc{process: MALDIS}
\end{flushright}
\end{document}