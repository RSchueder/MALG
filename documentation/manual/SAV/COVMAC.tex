\svnid{$Id: processes_library_um.tex 43694 2015-12-21 09:46:22Z markus $}
%------------------------------------------------------------------------------
\section{Vegetation coverage}
\begin{flushright}
\textsc{process: Coverage}
\end{flushright}

When a water body is covered by emerged macrophytes, the reaeration with oxygen and the light intensity in the water are decreased.
The parameter $fcov$ for the coverage of the surface water is an existing model coefficient, that is already used in the computation of the reaeration flux.
On the basis of the model for emerged macrophytes, the coverage can be computed.
The coverage with emerged macrophytes is assumed to be 100 \% when the actual emerged biomass reaches the maximum emerged biomass.
The parameter is used in the model equations for light intensity near the water surface.

\subsection{Implementation}

The growth of submerged vegetation can be limited by the underwater light climate. This depends on the dissolved
and suspended matter in the water (both organic and inorganic)
\cite{VanDuinEtAl2001} as well as on the shading due to emerged vegetation that covers the surface of
the water. The growth of emerged vegetation is not limited by light in this model, although several authors
report that selfshading can occur. In the light extinction function in \DWAQ, the coverage by emerged
vegetation is included via the following function:
%
\begin{equation}
Itop_1 = Is \times (1 - fcov)
\end{equation}
%
where:

\begin{tabular}{lll}
Itop$_1$   & light intensity at the surface of the water layer (layer 1)   & [W$\cdot$m$^{-2}$] \\
Is         & light intensity at the water surface                          & [W$\cdot$m$^{-2}$] \\
fcov       & fraction of the water surface covered by vegetation           & [-]                \\
\end{tabular}

The light intensity of the subsequent layers is computed according by the process CalcRad, see also
\Autoref{fig:lightUnderWater}:
%
\begin{align}
Itop_n = &Ibot_{n-1} \\
Ibot_n = &Itop_n \times e^{-ExvVl_n \times H_n}
\end{align}
%
where:

\begin{tabular}{lll}
ExtVL$_n$  & extinction of visible light in layer n          & [m$^{-1}$]         \\
Itop       & light intensity at the top of a water layer     & [W$\cdot$m$^{-2}$] \\
Ibot       & light intensity at the bottom of a water layer  & [W$\cdot$m$^{-2}$] \\
H          & thickness of the water layer, $z_{n-1} - z_n$   & [m]                \\
n          & index for water layers                          & [-]                \\
\end{tabular}

When there is only one water layer (compartment), the depth is equal to the water depth.

In the case that sediment layers are actually modelled the light intensities at the top and the bottom of these
layers are calculated in a slightly modified way:
%
\begin{align}
Ibot_n = & a \times Itop_n \times e^{-ExvVl_n \times H_n}
\end{align}
%
where:

\begin{tabular}{lll}
$a$          & amplification factor by scattering by sediment particles & [-] \\
\end{tabular}

\begin{figure}
\centering
\includegraphics[height=10cm]{Chapters/Vegetation/light_under_water.png}
\caption{The light intensity under water -- explanation of the variables in the light intensity functions.}
\label{fig:lightUnderWater}
\end{figure}

The coverage $fcov$ is calculated via:
%
\begin{equation}
fcov = \sum_i \frac{EM_i}{MaxEM_i}
\end{equation}
%
where:

\begin{tabular}{lll}
CoverageEM$_i$    & Coverage with emerged vegetation                       & [-]                 \\
CoverageSM$_i$    & Coverage with submerged vegetation                     & [-]                 \\
fcov              & Total coverage on the basis of all emerged species     & [-]                 \\
EM$_i$            & Actual biomass emerged vegetation                      & [g$\cdot$m$^{-2}$]  \\
SM$_i$            & Actual biomass submerged vegetation                    & [g$\cdot$m$^{-2}$]  \\
MaxEM$_i$         & Maximum biomass emerged vegetation                     & [g$\cdot$m$^{-2}$]  \\
MaxSM$_i$         & Maximum biomass submerged vegetation                   & [g$\cdot$m$^{-2}$]  \\
\end{tabular}
