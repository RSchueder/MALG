\svnid{$Id: processes_library_um.tex 43694 2015-12-21 09:46:22Z markus $}
%------------------------------------------------------------------------------
\section{Vertical distribution of submerged macrophytes}
\begin{flushright}
\textsc{process: MACDISii}
\end{flushright}

The actual location of the submerged biomass in the water column is species specific. Some species are evenly
distributed over the water column, where others tend to concentrate in the top layers.

The macrophytes biomass is administratively located in the bottom water layer. If the model is layered
(1Dv, 2Dv, 3D), the biomass is distributed over the vertical layers to
provide input for the modules that require such a distribution, in particular the vertical distribution of light
extinction.

\subsection{Implementation}

This process is implemented for different types of macrophytes, indicated by "ii" throughout this document.

The vertical distribution of the macrophyte biomass is calculated from the following input items
(for submerged macrophyte species ii):

\begin{tabular}{lll}
\textbf{Id in process} & \textbf{Symbol used} & \textbf{Description}                                             \\ [1ex]
Depth                & $D$                & depth of segment, layer thickness [m]                          \\
TotalDepth           & $T$                & total depth water column [m]                                   \\
LocalDepth           & $L$                & depth from water surface to bottom of segment [m]              \\
SMii                 & $M_T$              & submerged macrophyte biomass in water column [gC/m$^2$]        \\
SwDisSMii            &                    & type of macrophyte shape function (1: linear, 2: exponential)  \\
HmaxSMii             & $H_{max}$          & maximum height of macrohyte [m]                                \\
FfacSMii             & $F$                & parameter F in shape function [-]                              \\
\end{tabular}

\Autoref{fig:verticalDistribution} provides an overview of the geometrical quantities used in the calculation. Note that the
vertical co-ordinate $z$ is defined in a downward direction, with a value of zero at the water surface.

\begin{figure}
\centering
\includegraphics[height=8cm]{Chapters/Vegetation/vertical_distribution.png}
\caption{Definition of the quantities used for determining the linear vertical distribution.}
\label{fig:verticalDistribution}
\end{figure}

The vertical distribution of the submerged macrophytes \emph{m(z)} (in g/m$^3$) is represented either by a linear or
by an exponential function.

\subsubsection{Linear distribution (SwDisSMii=1)}
The shape function is defined by means of one dimensionless parameter $F$ with a value ranging from 0 to 2.
For $F = 0$, the biomass approaches zero at the top of the plant,
for $F = 1$ the biomass is distributed homogeneously and for $F = 2$, the biomass approaches zero near
the bed. Values between 0 and 1 result in a decreasing biomass from bottom to top. Values between 1 and 2 result
in an increasing biomass from bottom to top.

The biomass distribution $m(z)$ can be expressed by means of two constants $A$ and $B$, which are
formulated in terms of the total biomass $M_T$ and the shape parameter $F$:

\begin{align}
           m(z) &=  Az + B                                  \\
\nonumber  A    &=  \frac{M_T}{H_{max}} \frac{2-2F}{T- z_m} \\
\nonumber  B    &=  \frac{M_T}{H_{max}} \frac{F(z_m + T) - 2z_m}{T- z_m}
\end{align}

It is easy to derive that for $F = 1$, $A = 0$ and $B = M_T/H_{max}$.

Consequently, the algorithm to calculate the biomass in a layer from $z = Z_1$ to $z = Z_2$ proceeds as follows:

If $z_m > z_2$ (layer above top of plant): $M_{12} = 0$

If $z_m < z_1$ (layer entirely below top of plant): $M_{12} = \int_{Z_1}^{Z_2} m(z)dz$

Else (top of plant inside layer): $M_{12} = \int_{Z_m}^{Z_2} m(z)dz$

By integrating the mass distribution function we can derive the biomass in the layer, e.g.:
%
\begin{equation}
    \int_{Z_1}^{Z_2} m(z)dz = \frac{A}{2}(Z_2^2 - Z_1^2) + B(Z_2 - Z_1)
\end{equation}

The module calculates the following output items for submerged macrophyte species ii:

\begin{tabular}{ll}
FrBmSMii              &  Fraction of the macrophyte biomass in present layer [-]  \\
BmLaySMii             &  Macrophyte biomass density in present layer [gC/m$^3$]      \\
\end{tabular}

The fraction of the biomass in a layer from $z = Z_1$ to $z = Z_2$ is calculated as $M_{12} / M_T$.


\subsubsection{Exponential distribution (SwDisSMii=2)}

The exponential shape function is defined on an inverse vertical co-ordinate \emph{z'}, which is defined as
(see also \autoref{fig:verticalDistributionExp}):
%
\begin{equation}
   z' = T - z
\end{equation}

The value of $z'$ equals 0 at the bottom and it equals $H_{max}$ at the tip of the plant.

\begin{figure}
\centering
\includegraphics[height=8cm]{Chapters/Vegetation/vertical_distribution_exp.png}
\caption{Definition of the quantities used for determining the exponential vertical distribution.}
\label{fig:verticalDistributionExp}
\end{figure}

The mass distribution function is defined as follows:
%
\begin{equation}
   m(z') = A \cdot \bigl( e^{Fz'/H_{max}} - 1 \bigr)
\end{equation}

The shape function is defined by means of one parameter $F$. The constant \emph{A} is determined by
requiring that the integrated mass equals the total mass $M_T$. A value of $F$ approaching 0 defines a
linear distribution. Increasing values of $F$ define a stronger and stronger concentration of the biomass
near the plant tip (see \autoref{fig:examplesExpDistribution}).
%
The value of A can be determined as:
%
\begin{equation}
   A = \frac{M_T}{H_{max} \Bigl( \frac{{e^F} - 1}{F} - 1 \Bigr)}
\end{equation}

Consequently, the algorithm to calculate the biomass in a layer from $z' = Z'_1$ to $z' = Z'_2$ proceeds as follows:

If $Z'_1 > H_{max}$ (layer above top of plant): $M_{12} = 0$

If $Z'_2 < H_{max}$ (layer entirely below top of plant): $M_{12} = \int_{Z'_1}^{Z'_2} m(z')dz'$

Else (top of plant inside layer): $M_{12} = \int_{Z'_1}^{H_{max}} m(z')dz'$


By integrating the mass distribution function we can derive the biomass in the layer, e.g.:
%
\begin{equation}
\int_{Z'_1}^{Z'_2} m(z')dz' = A \cdot \Bigl( H_{max} \frac{e^{FZ'_2/H_{max}} - e^{FZ'_1/H_{max}}}{F} - (Z'_2 - Z'_1) \Bigr)
\end{equation}

\begin{figure}
\begin{subfigure}{0.3\textwidth}
\includegraphics[width=\textwidth]{Chapters/Vegetation/exponential_distribution1.png}
\caption{F = 0.001}
\label{fig:verticalDistributionExp1}
\end{subfigure}
\hfill
\begin{subfigure}{0.3\textwidth}
\includegraphics[width=\textwidth]{Chapters/Vegetation/exponential_distribution2.png}
\caption{F = 5}
\label{fig:verticalDistributionExp2}
\end{subfigure}
\hfill
\begin{subfigure}{0.3\textwidth}
\includegraphics[width=\textwidth]{Chapters/Vegetation/exponential_distribution3.png}
\caption{F = 20}
\label{fig:verticalDistributionExp3}
\end{subfigure}
\caption{Examples of the exponential vertical distribution for three values of the shape parameter $F$.}
\label{fig:verticalDistributionExp}
\end{figure}

%-------------------------------------------------------------------------------
\subsection{Hints}

The module uses a work array ("IBotSeg") which is filled during the first call.
This work array contains the segment number of the bottom segment that lies beneath each segment not
located in the bottom layer. This tells each segment where the biomass in the segment administratively
resides, as biomass can only exist in the bottom segment but is 'distributed' vertically in a post-processing step.
Note that in a 2Dh or 1D model this work array is trivial; every segment is the bottom segment of the whole water column.

For an exponential biomass distribution, the value of $F$ can range from a small positive number to
infinite. The table below provides the share of the biomass of the plant present in the upper 10 \% of the
plant height, as a function of $F$:

\begin{tabular}{lc}
\hline
$F$     &  Share of biomass in top 10\%  \\
\hline
0.1          &  19                            \\
2            &  26                            \\
4            &  36                            \\
6            &  46                            \\
8            &  55                            \\
10           &  63                            \\
15           &  78                            \\
20           &  87                            \\
25           &  93                            \\
30           &  96                            \\
\hline
\end{tabular}

The maximum allowable value for $F$ is 50 -- this is a numerical limitation, because otherwise the numbers could
become too large.

%This is an incorrect approach -- so removed from the documentation and the implementation
%
%In the present implementation, the input quantity maximum height of the plant $H_{max}$ is defined relative to the
%total water depth. Within the process module, it is converted to an absolute value based on the actual local
%value of TotalDepth.

\todo{Is the sentence below true indeed?}

In the present implementation, the overall biomasses are calculated as g/m$^2$.

