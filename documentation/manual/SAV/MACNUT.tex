\svnid{$Id: processes_library_um.tex 43694 2015-12-21 09:46:22Z markus $}
%------------------------------------------------------------------------------
\section{Uptake of carbon, nitrogen and phosphorus by macrophytes}
\begin{flushright}
\textsc{process: MACNUTSMii}
\end{flushright}

The phosphorus in the rhizomes is the first source for the growth of submerged vegetation. When the phosphorus
in the rhizomes is exhausted, the vegetation will switch to the uptake of phosphorus via the roots.

\DWAQ has several approaches to sediment modelling:
\begin{itemize}
\item
S1S2 simple sediment model
\item
SWITCH: a detailed four-layered model for nutrients in the sediment.
\item
DELWAQ-G: flexible number of sediment layers.
\end{itemize}

Currently only the "simple" sediment model S1S2 is considered. The S1S2 approach does not consider
dissolved nutrients in the pore water. The dissolved nutrients are released to the surface water immediately.
However, in the model for microphytobenthos \citep{DevelopmentMicrophytobenthos} a process called MPBNUT
can "intercept" the dissolved nutrients when they are released from decaying organic matter in the sediment.
One of the coefficients in MPBNUT is the thickness of the sediment layer, where microphytobenthos is mixed almost
homogeneously. The same implmentation is used for macrophytes.
%
\begin{align}
Nuptakesediment = &GrowthEM_i \times NCratEM_i - NtranslocRHtoEM_i +
                   GrowthSM_i \times NCratSM_i - NtranslocRHtoSM_i

\nonumber Puptakesediment = &GrowthEM_i \times PCratEM_i - PtranslocRHtoEM_i +
                             GrowthSM_i \times PCratSM_i - PtranslocRHtoSM_i
\end{align}
%
where:
\begin{tabular}{lll}
Nuptakesediment & Uptake of nitrogen from the sediment   & [g N$\cdot$m$^{-2}$]
Puptakesediment & Uptake of phosphorus from the sediment & [g P$\cdot$m$^{-2}$]
\end{tabular}

\subsubsection*{Hints for use}
The sediment should contain enough nutrients to support the growth of macrophytes. In this model, the growth
of macrophytes is NOT limited by a lack of nutrients in the sediment.

The release of dissolved nutrients depends on the decay of organic matter, containing nitrogen and
phosphorus. On the long run, the amount of organic matter in the sediment depends on the production
of organic matter in the lake. It is therefore possible that the nutrient pool in the sediment is
exhausted by the macrophytes.

\subsubsection{Sediment module DELWAQ-G}
\emph{TODO: Has this been done?}

The sediment module DELWAQ-G is a layered sediment model that considers dissolved nutrients. The
inorganic nutrients in the sediment are accessible for the roots of macrophytes. It is recommended
to link the macrophyte module to DELWAQ-G, too. The advantage of a layered sediment model is, that
nutrients will be stored in the deeper layers during the season. Roots affect the surrounding sediment,
because they can transport oxygen to the sediment.

DELWAQ-G and the other sediment modules will be considered next year.

